\documentclass{article}
\usepackage{multirow}
\usepackage{graphicx}
\usepackage{array}
\usepackage{algorithm}
\usepackage{algorithmic}
\usepackage{amsmath}
\usepackage{listings}
\date{today(2001,11,9)lilt al ra3ed}
\author{hammouali youcef}
\title{home work}
\begin{document}
\listoffigures
\listoftables
\listofalgorithms
\lstlistoflistings
\newpage
\begin{figure}
\maketitle
Dans ce devoir j'apprends à écrire en \LaTeX
Ce devoir sera noté comme assiduité, vous le faites individuellement. Je n'accepte pas les
binômes.
\section{Trafics logos}

\begin{center}
    

\hspace{0.7cm}
\includegraphics[height=3cm]{traffic.jpg} 
\hspace{0.5cm}
\includegraphics[height=3cm]{traffic1.jpg} 
\hspace{0.5cm}
\includegraphics[height=3cm]{traffic3.jpg} 
\end{center}
\caption{Exemple traffic logo}
\label{figl}
\end{figure}
\clearpage
\begin{table}
\caption{Table of types in maths}
\label{tab1}
\end{table}
\begin{table}
\begin{tabular}{|p{1cm}|p{3cm}|p{1.7cm}|p{4cm}|p{3.5cm}|}

\cline{2-5}  
\multicolumn{1}{c|}{.} & \multicolumn{2}{c|}{Type 1}  &  \multicolumn{2}{c|}{Type 2}  \\
\hline
1 & $Deflection = \omega = 0$  & Y(0)  T(t) = 0 & $ Slope = \frac{d}{dx}\omega = 0$ & (L,t) = m0 . $ \frac{d^{2}}{dt^{2}\omega}(L,t) $ \\
\hline
2 & $ Slope = \frac{d}{dL}\omega = 0$ & Y'(L) T(t)= 0 &  4.E.I.Y'''(L).T(t) = m0.Y(L).Ï(t) & 4.E.I.$ \frac{d^{3}}{dx^{3}}\omega = 0 $ \\
\hline
\end{tabular} 
\end{table}
\section{Table maths}
\section{Équations mathématiques}
\section*{Equation 1}
\begin{center}
$ P_{xi} = \overline{U_{x}}+\sigma_{x}\frac{\Sigma^{N_{u}}_{k}D_{kx} x (\frac{S_{ki}-\overline{U_{k}}}{\sigma_{k}})}{\Sigma^{N_{u}}_{k}D_{kx}} $
\end{center}
Where : \newline

$P_{xi}$: is the predicted rate for user x on item i \newline

$S_{ki}$: is the rate of song i given by user k \newline

$\overline{U_{x}}$: is the average rate of user x \newline

$\overline{U_{k}}$: is the average rate of user k \newline

$\sigma_{x}$: is the standard deviation of all the rates of user x
\newpage
\section*{Equation 2}
Alternatively we may write the last display as: \newline

$\displaystyle\left\lvert \int^{b}_{a} f(t) - g(t)\vert =  \lim_{n \to \infty} \sum_{i = 0}^{n} M_{i}(t_{i}-t_{i-1)} \right\rvert$
\begin{center}
$\leq \sum_{i = 0}^{n} M_{i}(t_{i}-t_{i-1)} = \int^{b}_{a} \lvert f(t) - g(t)\rvert dt < \delta = \varepsilon  $
\end{center}
L'apparence du texte normal x + y = 3 et celle des
formules mathématiques x + y = 3 peuvent diérer.
\begin{center}
$ t = \sqrt{\dfrac{ln(c}{-2\lambda\sqrt{c}}} $ \qquad  $ f(x) = \dfrac{1}{\sqrt{2\pi\sigma}}  {e^{-\dfrac{(x-\mu)^{2}}{2\sigma^{2}}}}$ \newline

$ H(X) = -\sum{i = 0}P(x_{i)log_{b}P(x_{i})} $
\end{center}

\section{linsting de latex}
Dans cette partie faire un include d'un chier de n'importe quel fichier \LaTeX que vous avez deja avec la methode \textbf{inputlisting} 


\begin{lstlisting}[]
\end{lstlisting}




\begin{lstlisting}[caption={Listing 1: code latex}]

\documentclass{article}
\usepackage{algorithm}
\usepackage{algorithmic}

\begin{document}
\begin{algorithm}
\caption{Algo 1}
\end{lstlisting}
\newpage
\begin{lstlisting}
\begin {algorithmic}[1]

\STATE $y \leftarrow1$
\ IF{$n < 0$}
\STATE $X \leftarrow1/ x$
\STATE $N \leftarrow -n$
\ELSE
\STATE $X \leftarrow x$
\STATE $N \leftarrow n$
\ENDIF

\ end{algorithmic}
\ end{algorithm}

\end{document}
\end{lstlisting}
\section{Algorithmique}
\newpage
\begin{algorithm}

\caption{Algorithm converts SOP equation to RPN}

\begin{algorithmic}[1]

\REQUIRE Initialize a $stack$ and a $queue$

\FOR{$token$ in $SOP$}

\IF{$token$ is a label}

\STATE add $token$ to $queue$

\ELSIF{$token$ is + or *}

\IF{$token$ is + and top of $stack$ is *}

\STATE pop the top of $stack$ to $queue$

\ENDIF

\STATE push $token$ onto $stack$



\ELSIF{$token$ is (}

\STATE push $token$ onto $stack$

\ELSIF{$token$ is )}

\STATE keep popping operators from $stack$ to $queue$ until sees (

\ELSE

\STATE error, exit

\ENDIF

\ENDFOR

\STATE No more $token$ to consume, pop operators from $stack$ to $queue$

\RETURN $queue$

\end{algorithmic}

\end{algorithm}




\end{document}